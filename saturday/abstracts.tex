
\abstractwithspeaker{09:30}{Building B}{Victor Olaya}{Versioning OSM data with GeoGit}%
{Decentralized versioning for OSM}%
{Victor Olaya is developer at Boundless, QGIS developer, and the main author of the QGIS processing framework.}%
{GeoGit is a decentralized versioning tool for geospatial data. This talk introduces GeoGit and discusses some of its OSM-specific features. These features bring the advantages of decentralized versioning to OSM and improve the management of OSM data.}

\abstractwithspeaker{09:30}{Building A}{João Porto de Albuquerque}{Identifying Elements at Flood Risk with Volunteered Geographic Information}%
{An Approach based on OpenStreetMap with a Case Study in Cologne}%
{João Porto de Albuquerque is visiting Professor at Heidelberg University and Professor at University of Sao Paulo, currently researching on Collaborative Systems and VGI for Disaster Risk Management. }%
{The identification of elements at risk is an essential part in Hazard Risk Assessment. Especially for recurring natural hazards like floods, an updated database with information about critical elements at flood risk (e.g. schools, hospitals etc.) is fundamental to support crisis preparedness and response activities. However, acquiring and maintaining an up-to-date database with elements at risk requires both detailed local knowledge and hazard-specific constraints, being often a challenge for local communities due to lack of expertise and appropriate funding, deferred priorities and complex political determinations.

This talk presents a new approach for leveraging OpenStreetMap to automatically identify hazard-specific elements at risk. We provide a new data model for extracting elements at risk from OSM and conduct a case study in the city of Cologne, Germany to validate our approach.}

\abstractwithspeaker{10:00}{Building B}{Oliver Tonnhofer}{Imposm}%
{The Other PostGIS Import Tool}%
{Oliver Tonnhofer is the lead developer of Imposm and MapProxy. He is also co-founder of Omniscale, a German IT firm specialized in fast geographical maps and the acceleration of existing geospatial data infrastructure.}%
{Imposm is an open source tool that imports OpenStreetMap data into PostGIS databases.

Imposm is fast and flexible. It supports custom database schemas and can generalize complex geometries for efficient rendering.
The presentation will show you how Imposm works and in which ways it differs from osm2pgsql. You will also learn about the status of the upcoming Imposm 3 release, which will feature incremental update support and performance improvements.}

\abstractwithspeaker{10:00}{Building A}{Francesca Murtas }{SardSOS: More than a Map}%
{An Emergency Call. When Mappers Go United\dots}%
{Francesca Murtas is an interaction designer. Mapping is a state of mind.}%
{I would like to share and talk about my experience with the free map SardSOS that I've created while confronting with the devastating floods which damaged my island, Sardinia, in November 2013. I've built an open crowdmap on the Ushahidi platform (http://sardsos.crowdmap.com/) which has been visited by nearly 14,000 people in the first week.

This project has also triggered the release of open geo-data from the Sardinian government. It pushed the Italian mainstream media to talk about OSM and inspired other regions and public administrations towards opening their data and looking at OSM with interest.

I will also report about an independent online research, after the emergency, aimed at checking the use and the perception of people of OpenStreetMap and mapping for emergencies.}

\abstractwithspeaker{11:00}{Building B}{Roland Olbricht}{Sparse Editing}%
{Editing Large Scale Objects}%
{Roland Olbricht is the developer of the Overpass API.}%
{While OpenStreetMap has an impressive data quality on small scale objects, large scale objects are often only mapped in low quality. One example is the often discussed subject of mapping public transport services, and indeed these are far from complete or consistent.

The deeper reason for this is that editing large scale objects is exceptionally hard. This results from the fact that the larger an object is, the more spatial dependencies it is involved in. For example, a public transport relation can be damaged by each and every roundabout or lane mapping attempt.

In this talk we will analyze which typical kinds of dependencies exist and discuss various approaches like route suggestion by involving a route engine or mapping on a filtered subset of data. As a hands-on example it is shown how to efficiently recover a lost name tag on the river Rhine by editing an area of more than a hundred kilometers extent in JOSM.}

\abstractwithspeaker{11:00}{Building A}{Harald Koertge}{Efficient Routing for Mobile Systems}%
{Focussing on OSM and Comparing Against Routing with Professional Data}%
{Harald Koertge founded his first company developing navigation systems in 1997 for PalmOS devices. Since that time, he has developed and designed multiple generations of map data models, routing engines, etc. for commercial and OSM data.}%
{As OSM is gaining more traction for usage in navigation and routing for mobile devices and applications, it’s important to understand the differences between commercial map data and OSM data, particularly for effective route calculations. 

Following a brief history of routing engines that were primary developed for high performance usage on mobile devices, I will share what and where the challenges are in OSM versus commercial/professional data, from the perspective of having been entrenched in both worlds for many years.}

\abstractwithspeaker{11:30}{Building B}{Simon Legner}{The JOSM Editor}%
{Current Development and Data Validation with MapCSS}%
{Simon Legner has been a JOSM developer since 2011, an OSM mapper since 2008, and is a computer scientist.}%
{JOSM is a Java-based offline editor for OpenStreetMap which has been around since 2005. A recurring challenge for editors is validating user input, guiding mappers towards making as few mistakes as possible.

This talk will provide insights in the current development. Emphasis will be on the use of the MapCSS styling system for data validation, a technique which has been developed in recent months.}

\abstractwithspeaker{12:00}{Building B}{Oliver Kaleske}{Wall·E}%
{An OpenStreetMap Bobot}%
{Oliver Kaleske is a physicist, working in software development, and has been an OSM contributor since 2010.}%
{Wall·E is a maintenance robot I have been operating on OpenStreetMap in Germany since the end of 2012. The edits performed range from simple tasks such as the reduction of repeatedly referenced nodes in
ways to relatively sophisticated corrections in addr:* tags. Compared to other robots, Wall·E still has a relatively low total edit count (roughly 30k by early 2014), but this also
reflects the rather conservative editing strategy employed, which includes several precautions to avoid misfixes.

The talk will cover some of the technical aspects (filtering patterns, tools involved, safety measures) as well as the robot's history starting from xybot, an earlier OSM robot which served as
a model for many of Wall·E's functions, and some remarks on the role of robots within the social system of OSM.}

\abstractwithspeaker{14:00}{Building B}{Tim Teulings}{Introduction to libosmscout}%
{A C++ Library for Offline Rendering and Routing}%
{Tim Teulings has been an open source and professional software developer for more than 25 years (now 43 years old). He has developed a number of open source products over the years, most with a small user base. He has been working on libosmscout since around summer 2009 with an increasing user base and even some code contributions.}%
{Libosmscout is a C++ library for offline map rendering, location look-up and routing. It targets mobile devices as well as the desktop, and is is highly customizable.

The talk introduces you to the project by showing potential use cases, presenting main features, giving a few simple code examples and some technical background. Discussion is expected and encouraged.}


\abstractwithspeaker{14:00}{Building A}{Kirill Bondarenko}{OSM: World Map or Set of Local Maps?}%
{Pecularities of national mapping}%
{Kirill Bondarenko is OSM editor since 2009. His primary interest is usage of OSM data in navigation software. He also maintains some validatation tools for the Russian community.}%
{Each country has peculiarities, and osm data is no exception. It becomes a big problem, if you need osm-based map of the world, e.g. routable map for PNA.

I will tell about local mapping rules and techniques, mainly about Russia in comparison with Europe. Tagging of common objects, address schemes, road classification, etc.}


\abstractwithspeaker{14:30}{Building B}{Thomas Graichen}{A Combined In- and Outdoor Map Application for Android}%
{A seamless in- and outdoor map viewer for OpenStreetMap data and its implementation}%
{Thomas Graichen studied information engineering at the Technische Universität Chemnitz. He is currently member of the scientific staff of the Chair for Circuit and System Design, interested in hiking, biking and therefore in OSM outdoor maps, too.}%
{Although the OSM community has realized several indoor projects, an Android application for viewing indoor maps remains unavailable. As part of an electric mobility project, the group at TU Chemnitz has developed an application that fills this gap.

The concept and an overview of the chosen implementation shall be presented here. The talk will focus in particular on the usage of the mapsforge library, which is widely used for creating outdoor maps, and how it was applied to draw indoor maps. The indoor data itself is described with the scheme proposed by Marcus Goetz (http://wiki.openstreetmap.org/wiki/IndoorOSM).

Furthermore future developments and possible usage scenarios for this map application are presented.}


\abstractwithspeaker{14:30}{Building A}{Daniel Kastl}{Where the Streets have no Name}%
{Mapping in Japan}%
{Daniel Kastl is geographer, mapper, software developer, open source and open data advocate. He was born in Germany but is currently living in Japan.}%
{From a European or North American perspective many things seem to be clear, but what you take as granted in a ``Western'' country is likely to be different somewhere else.
This talk gives an example, how a street-based address schema can become a real headache in countries which don't know street names.}

\abstractwithspeaker{14:50}{Building A}{Dirk Helbing}{Keynote: How to Create a Better World}%
{Planetary Nervous System, Global Participatory Platform, Social Information 
Technologies: How to Create a Better World}%
{Dirk Helbing is Professor of Sociology, in particular of Modeling and 
Simulation, and member of the Computer Science Department at ETH Zurich. 
He earned a PhD in physics and was Managing Director of the Institute of 
Transport and Economics at Dresden University of Technology in Germany. 
He is internationally known for his work on pedestrian crowds, vehicle 
traffic, and agent-based models of social systems. Furthermore, he 
coordinates the FuturICT Initiative (http://www.futurict.eu/), which 
focuses on the understanding of techno-socioeconomic systems, using Smart 
Data. His work is documented by hundreds of scientific articles, keynote 
lectures and media reports worldwide. 

Helbing is elected member of the World Economic Forum's Global Agenda 
Council on Complex Systems and of the prestigious German Academy of 
Sciences "Leopoldina". He is also Chairman of the Physics of Socio-Economic 
Systems Division of the German Physical Society and co-founder of 
ETH Zurich’s Risk Center. In 2013, he became a board member of the 
Global Brain Institute in Brussels, and in 2014 he received a honorary 
PhD from the TU Delft.}%
{It probably started with Linux, then came Wikipedia and Open Street Map. 
Crowd-sourced information systems
are central for the Digital Society to thrive. So, what's next? 
I will introduce a number of concepts such as
the Planetary Nervous System, Global Participatory Platform, 
Interactive Virtual Worlds, User-Controlled
Information Filters and Reputation Systems, and the Digital Data Purse. 
I will also introduce ideas such as the Social Mirror, 
Intercultural Adapter, 
the Social Protector and Social Money as tools to create a better world. 
These can help us to avoid systemic instabilities, market failures, 
tragedies of the commons, and exploitation, and to create the framework 
for a Participatory Market Society, where everyone can be better off.}


\abstractwithspeaker{16:00}{Building B}{Cristian Consonni}{Creating a bridge between OpenSteetMap and Wikipedia: Wikipedia-tags-in-OSM}%
{A tool to add Wikipedia tags in OSM and coordinates in Wikipedia}%
{Cristian Consonni is researcher at the "Digital Commons Lab" unit of the Fondazione Bruno Kessler (FBK), in Trento, Italy, Wikimedia Italia's vicepresident and Wikipedian, free software activist, physicist and storyteller.}%
{When you visit a Wikipedia article for a monument or a place (e.g. the Colosseum) you can find a link which will display the same object highlighted on OpenStreetMap: this tool is called WIWOSM and it was created by German mapper and Wikipedian Kolossos. It works using the "Wikipedia" tags, i.e. wikipedia=language:article, added by volunteers
in OSM.

This presentation introduces a new tool called Wikipedia-tags-in-OSM (WTOSM): a script producing a set of web pages that makes easier to add the "Wikipedia" tags in OSM using JOSM "remote control" feature, and, at the same time makes easy adding coordinates in Wikipedia articles using the {{Coord}} template and OAuth authentication. This tool has been developed by user Groppo, with a great involvement by
the Italian OSM and Wikipedia communities.

In this presentation I will present WTOSM, which is available online at http://bit.ly/wikipedia-tags-in-OSM, and its features and how it
can be used also to map places when the only information available is the Wikipedia article abstract, using Nuts4Nuts (presented at SotM13,
see http://bit.ly/presentation-Nuts4Nuts-SOTM). This project is realeased as free software (GPLv3) and its source code is available on
github.

We believe that this tool can help the OpenStreetMap community to discover new objects to map from Wikipedia pages and also it can
create a bridge among the two projects.}

\abstractwithspeaker{16:00}{Building A}{Peter Neubauer}{Mapillary -- the missing view of the planet}%
{An Update on the State of the Street}%
{Peter is co-founder of open source projects such as Neo4j, OPS4J and Qi4j. Right now, Peter is concentrating on starting up Mapillary.com -- a project to provide a crowd sourced, public (street) view of the planet with smartphones. Peter is a mentor helping startups at Startupbootcamp Copenhagen, Berlin and at Mozilla WebFD. He likes teaching programming to kids at CoderDojo Malmö and organizing events like http://www.thoughtmade.com and TEDx Öresund and LAN-parties for kids like kidscraft.se.}%
{We want to create a photo representation of the world, a map with photos of every place on Earth. Current street view alternatives have so many limitations from the fact that they are created using cars with camera rigs, focusing on streets. There are many places on this planet that they will never get to, and even the places they do get to will not get updated as often as you might want them to. Together we can fix that. Also, Mapillary wants to provide the image data to OSM for enhancing that data. We need help in choosing the right license for the images! Peter is going to talk about the technical and some of the legal aspects of the system and give some insights into the possible use cases and some existing (after just a few months) interesting usage.}

\abstractwithspeaker{16:30}{B}{Alfonso Crisci}{Data consistency in OpenStreetMap}%
{Monitoring consistency using spatial features and tag semantics}%
{Alfonso Crisci is native biometerologist/geostatistician going to OSM-ollywood with some crazy ideas.}%
{Monitoring consistency and reliability of maps is an important task when Volunteered Geographic Information (VGI) is involved and OpenStreetMap provides a good testing ground to develop operative methodologies that can be used as a web application. Two questions about data consistency are considered: for any given region (I) Is the level of spatial features density at a given scale enough for a suitable geographical description? (II) Is the semantics of the features (described by keys and tags) consistent and comprehensive?

This talk presents preliminary results from work done by Alfonso Crisci (IBIMET-CNR), Maurizio Napolitano (FBK-Trento), Francesca De Chiara (FBK-Trento), Valentina Grasso (IBIMET-CNR, LaMMA Consortium), and Cristian Consonni (FBK-Trento).}

\abstractwithspeaker{16:30 (geocoding session)}{Building A}{Gary Gale}%
{Geocoding---the Missing Link for OSM?}%
{Geocoding Isn't Easy and yet It's Vital to the Success of OSM}%
{A self-professed map addict, Gary has worked in the mapping and 
location space for over 20 years through a combination of luck and 
occasional good judgement. Currently consulting with Lokku as 
Geotechnologist in Residence, Gary is helping to advance open 
geospatial technologies and bring them to new markets. 
%A Fellow 
%of the RGS, he tweets about maps, writes about them and even makes them. 
}%
{
%So many uses of maps need some form of geocoding. 
% Geocoding is 
% simple, just strip out known stop words and look up the place or 
% address in a gazetteer and hey presto, instant coordinates. The 
% same applies for reverse geocoding. Right? But it's 
% not easy and trivial. While OSM continues to make massive 
% strides towards being the definitive source of open mapping data, 
% geocoding has ... languished a bit.
This talk will tell you why 
geocoding isn't easy. 
It will tell you why there are differences in geocoding between the countries
% It will tell you about why once you can geocode in one country it 
% doesn't mean you can automatically geocode in another country. 
% It 
%will also tell you about 
and why Nomimatim isn't the only open source 
geocoder 
%that's out there. 
% And this talk may have something exciting 
% from Lokku and OpenCage Data to announce, which may just be related 
% to geocoding. 
%Geocoding shouldn't be the missing link for open maps 
%and for OSM; this talk will tell you why. 
}

\geocodingabstract{16:30 (geocoding session)}{Building A}{Randy Meech}{Pelias}%
{A new OpenStreetMap geocoder with Elasticsearch}%
{Randy is CEO of Mapzen, a company working on an open source mobile mapping application. In a prior role at MapQuest, he worked to re-orient the company toward open data and tools}%
{%
% Users of mobile mapping applications expect a forward geocoder with 
% autocomplete and flexible search results, as well as an accurate 
% reverse geocoder. 
Recent versions of Elasticsearch offer both 
excellent geo support and a fast completion suggester. Pelias is a 
new open-source project with tools for indexing 
OSM, Geonames, and Quattroshapes data into Elasticsearch, and a 
simple server to handle queries and auto-complete suggestions. 
A major goal of the project is to make it easy to adapt the query parser 
and data pre-processors to the requirements of specific languages and 
locations. For example, users in New York City might require searchable 
intersections, while French-speaking users in Quebec might need a 
different query parser from those in France. 
% It should be easy to edit 
% these rules such that little programming experience is necessary. 
This 
presentation will cover the basic architecture and data strategies of the 
project, provide some details into Elasticsearch's geo and completion 
support, and will have a big focus on how to contribute. 
%Demo: http://mapzen.com/pelias Github: https://github.com/mapzen/pelias
}

\geocodingabstract{16:30 (geocoding session)}%
{Building A}{Dmitry Kiselev}{OSM Gazetteer}%
{OSM Gazetteer or What's Wrong with Nominatim}%
{Dimitry has been OSM Contributor since 2010 and is a keen GIS programmer. }%
{This talk is about big search engines and osm, local addressing traits,
Sphinx/Solr (Lucene) and OSM, and PostGIS or not PostGIS. }

\abstractwithspeaker{17:00}{Building B}{Jerry Clough}{Beyond the 3 "R"s}%
{The challenges of using OpenStreetMap data for analysis}%
{Jerry Clough has been interested in maps since the age of 4. He has a professional background in scientific research (Genetics, Computer Science) and business consultancy, and is an enthusiastic amateur naturalist. OSM forms a natural nexus between these diverse interests. }%
{Most uses of OSM data fall into 3 categories, the 3 "R"s: Rendering (cartography), Routing, and 'Rummaging' (search, geolocation). However, the large and diverse sets of data within OSM also have considerable, and under-appreciated potential for answering analytical questions.

The patchiness and lack of completeness of OSM data significantly hinders its use for analysis.  But there are other aspects of the data which don't help either.

Examples of how OSM data can be used for analysis will be presented to demonstrate both the potential and the underlying issues.

Analysis places different demands on how data is mapped in OSM: a focus on using specific subsets of the data will highlight inconsistencies, and identify missing information. Furthermore it is often the case that data that is notionally derivable from OSM is not so in practice. 

Making OSM data usable for analytics tests and stresses how the data are mapped in ways which are quite different from the typical uses. Therefore particular analysis problems can help enrich and extend how and what we map.}
