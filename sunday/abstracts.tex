% abstract sunday
\newcommand{\sundayabstract}[7]%
{%
\newpage%
\ifthispageodd{\ThisCenterWallPaper{1.0}{sunday_r}}{\ThisCenterWallPaper{1.0}{sunday_l}}
\renewcommand{\talktime}{#1}
\renewcommand{\talkroom}{#2}
\thispagestyle{scrheadings}
\noindent \emph{#3}%
\vspace{0.75em}
{\par\noindent\large \sectfont #4}%
\vspace*{0.35em}%
{\par\noindent\bfseries \normalsize \sectfont #5}
\vspace{1em}
\par\noindent #7 \par%
\vspace*{0.35em}%
{\em{#6}}%
\cropmarkswallpaper%
}

\ThisCenterWallPaper{1.0}{excursion}
\sundayabstract{8:00}{Railway Station}{Jerry Clough}{Woodland and Wetland Mapping}%
{Sunday Morning Excursion}%
{}%
{This is an opportunity for interested mappers to work together in the
field in a variety of woodland and wetland (reed-beds, open water, marsh, etc.) habitats which are of interest to participants and discuss how to recognise and distinguish the features of those habitats in such a way that they can be mapped. 

The anticipated destination is nature reserve Weingar\-tener Moor/Bruchwald Grötzingen, which is about 10\,km NE of Karlsruhe, and easily reached by a short train journey from Karlsruhe Hbf. We will meet at Karlsruhe Hbf at 08:00 in time for the 08:10 train to Weingarten (Baden), Meeting point at the entrance to the platforms. The nature reserve is around 1 km walk from Weingarten Bhf. We will spend between 2.5--3 hours in the nature reserve. Return train at either 11:57 or 12:36.

Participants are advised to bring appropriate footwear, sunscreen/rain gear, mosquito-repellant, and something to drink. Binoculars, hand-lens and tree and plant field guides will also be useful, although these will be available to consult or borrow.

Follow \emph{@SK53onOSM} on Twitter for updates and last-minute changes.
}

\sundayabstract{10:00}{Amalienstraße}{Jochen Topf}{Osmium to the Rescue}%
{Solving OSM Problems with Osmium}%
{Jochen has been an active OSM contributor and developer of OSM software since 2006. He is co-author of a book about OSM (http://www.openstreetmap.info). He has turned his OSM hobby into a business developing software and as consultant on all things OSM and geo. More at http://jochentopf.com/ . }%
{Osmium is a highly flexible and performant C++ library for working with OpenStreetMap data. Built on top of this library are a command line tool and a Node.JS module. This suite of software can be used for many tasks, from keeping history planets current, to creating statistics, to converting OSM data in many different GIS formats. This workshop presents several typical problems from the OpenStreetMap world and shows how they can be solved using the Osmium library, the Osmium command line tool, or the Node.JS module. 

You can find more about Osmium and related software at http://osmcode.org/. }

\sundayabstract{11:00}{Amalienstraße}{Roland Olbricht}{Overpass API v0.7.50}%
{The Temporal Dimension}%
{Roland has been a mapper since 2008. He started Overpass API in 2009 to make OpenStreetMap data more accessible. He has a PhD in pure mathematics, a strong background in computer science, and a passion for high performance algorithms and transport networks. Current\-ly he works as a computer scienst at Fraunhofer SCAI in Bonn. }%
{With OpenStreetMap getting more mature, it becomes more important to track changes. But doing so is difficult for various reasons: Changeset comments can be misleading, destruction may be hidden in innocent looking changes, and the sheer amount of data prohibits a complete manual review. For that reason Overpass API provides, from version 0.7.50 on, the complete history since the license change. 

In this workshop we'll demonstrate the commands to access that history. We explain how these features allow building powerful client JavaScript only websites that can track changes with arbitrary search criteria. We will discuss what remains to be done to build a complete and logically consistent revert tool based on Overpass API v0.7.50. }


\sundayabstract{14:00}{Amalienstraße}{Andy Allan}{OpenStreetMap Carto}%
{The State of the Style Sheets---One Year On}%
{Andy is a freelance digital cartographer and open sour\-ce geospatial developer. He has been creating maps from OSM data since 2007 and started the OSM Carto project in November 2012. }%
{ In this workshop we will review our progress over the
last 12 months, lay out the future roadmap and look at
some of the interesting projects that have used OpenStreetMap Carto as a foundation. We will also show you how to
customise the stylesheets for your own projects using TileMill, and
show you how to contribute improvements back to the main map style.  }

\sundayabstract{15:00}{Amalienstraße}{Martijn van Exel}{MapRoulette Next Generation}%
{User Defined Challenges}%
{Martijn's bond with OpenStreetMap dates back to 2007, when he attended his first mapping party in Amsterdam. He was hooked and became a community leader in the Netherlands. He continued this role when he moved to the United States in 2011. He has been serving on the United States Chapter board since then - the last two years as president. Martijn works at Telenav, a global provider of personal navigation software, as an OSM expert. }%
{In this workshop, you will learn how to get your own challenges into MapRoulette, the OpenStreetMap QA tool. We will discuss the key elements of a good challenge, how to prepare your challenge for MapRoulette, updating strategies and more.
}%
