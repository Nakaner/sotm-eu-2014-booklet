% typeset an abstract without speaker name
% DEPRECATED -- DO NOT USE ANYMORE
\newcommand{\talkabstract}[4]%
{%
\newpage%
\subsection*{#1}%
\subsubsection*{#2}%
#4 \par%
{\em{#3}}%
}

% typeset an abstarct with speaker name
% REPLACES \talkabstract
% Please attend that the spacing between the bold lines may change after every 
% compilation if the text is longer than one page!
%
% USAGE:
% \abstractwithspeaker{Joe Average}{title}{subtitle}{speakers's bio}{abstract text}
\newcommand{\abstractwithspeaker}[7]%
{%
\newpage%
\renewcommand{\talktime}{#1}
\renewcommand{\talkroom}{#2}
\thispagestyle{scrheadings}
\noindent \emph{#3}%
\vspace{0.75em}
{\par\noindent\large \sectfont #4}%
\vspace*{0.35em}%
{\par\noindent\bfseries \normalsize \sectfont #5}
\vspace{1em}
\par\noindent #7 \par%
\vspace*{0.35em}%
{\em{#6}}%
}

\newcommand{\geocodingabstract}[7]%
{%
\vspace{2em}
\renewcommand{\talktime}{#1}
\renewcommand{\talkroom}{#2}
\thispagestyle{scrheadings}
\noindent \emph{#3}%
\vspace{0.75em}
{\par\noindent\large \sectfont #4}%
\vspace*{0.35em}%
{\par\noindent\bfseries \normalsize \sectfont #5}
\vspace{1em}
\par\noindent #7 \par%
\vspace*{0.35em}%
{\em{#6}}%
}

\newcommand{\talkabstractwithoutsub}[3]%
{%
\newpage%
\subsection*{#1}%
#3 \par%
{\em{#2}}%
}

\abstractwithspeaker{10:00}{Building B}{Dennis Luxen}{Everything but directions.}%
{All the other exciting things you can do with routing.}%
{Dennis Luxen, the lead developer of Open Source Routing Machine, a high-performance routing engine, is an algorithm engineer at heart. He holds a MSc as well as a PhD in computer science. His work focuses on highly scalable route planning mapping services. Dennis is currently working at Mapbox. }%
{The talk explores the benefits of a state-of-the-art routing engine that go beyond the means to get simple driving or walking directions. We explore how 
routing is used as a building block not only to sophisticated data analysis but also in other new and exciting areas. Whether it is applied to calcaluate the walkability of neighborhoods, the matching of riders and drivers for carpooling, or even the monitoring for vandalism in the OpenStreetMap database, 
among a couple of further use cases. And in many cases these questions can even be answered in real-time without any noticeable delay, delivering a superior performance.

With the challenge of handling an ever-growing data set, it is important to use methods that handle data efficiently and can keep up with the data growth and change rates. The talk shows the combination of algorithm engineering, originated in the fundamental research of academia, and the crowd-sourced data 
of OpenStreetMap delivers not only superior features but also a superior user experience.}

\abstractwithspeaker{10:00}{Building A}{Ilya Zverev }{I've bought a car for mapping, now what?}%
{First: glue an "OpenStreetMap" sticker to it}%
{Long time mapper who likes to go outside, editor of SHTOSM, a russian OSM news blog, organizer of some mapping parties and conferences. }%
{The author focuses on field mapping: collecting data on the move, both on passenger seat with a notebook, and behind the wheel, when you cannot spare a second to look away from the road. There will be a history of attempts at mapping as much as possible, sometimes with large-scale projects and examples of what commercial providers can do. Entering data in OpenStreetMap is also a major problem: it has to be easy to be done on large scale. The topic affects mapping on feet and on bicycle, since the concepts developed will affect field mapping in general. Finally, the main question: do you really need a 4x4 jeep for mapping Russia?}

\abstractwithspeaker{11:00}{Building B}{Marco Quaggiotto }{BikeDistrict - The smart city by bicycle}%
{Crowdsourced tools to support urban cyclists and urban bike planning}%
{\emph{The District} is a company based in Milan specialized in digital services for urban mobility. BikeDistrict it is the the main project (www.bikedistrict.org) available for web and mobile, is an interactive OSM-based map for those urban cyclists riding in "not entirely" bike-friendly cities.}%
{In the context of the optimization of resources to devote to bicycle mobility, the two-years experience of the BikeDistrict project will be presented.
BikeDistrict is primarily a web and mobile application working as a free navigation tool, suggesting the urban bike-friendly paths. 
An intelligent "street rating" system has been developed for the BikeDistrict map interface with the double purpose of:
\begin{itemize}
\item allowing the complete customization of the bike path, according to the specific needs of the user,
\item contemporarily collect and aggregate users "street ratings" to better inform the routing algorithm, and the bike network database, with the final aim of providing the best possible routes for the different categories of urban cyclists. 
\end{itemize}
 
BikeDistrict is then configured as an urban bike mobility monitoring and analysis device. BikeDistrict has the ability to identify critical infrastructural gaps, to evaluate the potential benefits of specific interventions on the bicycle network and in general to support the planning activities by providing real-time-data and analysis related to the infrastructure condition and demand patterns. BikeDistrict is capable of collecting valuable information about the bike mobility demand (e.g 400 000 origin/destination itineraries in the city of Milan during 2013) and regarding the perception of quality of the road infrastructure and the bike facilities, expressed directly by the bike users. }

\abstractwithspeaker{11:00}{Building A}{Thilo Stapff, Johannes Bouchain }{Opengeofiction}%
{Using the OSM software for mapping a fictional planet}%
{Thilo Stapff has always been fascinated by maps, real or imaginary. 
He studied Mathematics and works as software developer in Frankfurt. He created
Opengeofiction in 2013. Johannes Bouchain, Hamburg, is interested in
(imaginary) cartography since childhood, Web designer/urban planner and
Opengeofiction co-founder.}%
{Opengeofiction (http://ww.opengeofiction.net) is a collaborative platform for the creation of fictional maps, founded in 2013 using the OSM software stack. In this presentation, we want to describe our motivation for working with fictional maps, our fictional mapmaking history, how and why we created Opengeofiction, and how the project is currently developing. We will also highlight some of the technical challenges we encountered.

Employing the OSM software has taken our fictional mapmaking to a whole new level, not only for practical reasons, but also because it has allowed us to transform a somewhat solitary hobby into a collaborative activity within a growing community. While at a first glance Opengeofiction might look like a miniature version of the Openstreetmap project itself, there are also some key differences. We will explain why we consider the OSM software stack to be generally very well suited for a project like ours, but also discuss the areas where it does not quite fit as well.

We're looking forward to present the Opengeofiction project on the SotM-EU 2014 conference and we hope that we'll give an interesting input about how the OSM software can be used in a rather unconventional way.  And we will show you some far away countries where you’ve never been before.}


\abstractwithspeaker{11:30}{Building B}{Thomas Jakubicka }{The use of OpenStreetMap data in journey planning}%
{Example on the usability of the data for railwayrouting}%
{Thomas Jakubicka is responsible for project management at Mentz Datenverarbeitung. He is especially involved in many projects related to OSM.}%
{The continually improving accuracy, grade of detail and completeness in OSM data on the one hand and the always rising demand for data with higher information density, has drawn the attention of public transport operators and autorities towards OSM. Furthermore open governmental data becomes more and more important to public transport services and the idea of OSM supplements this approach.
Mentz Datenverarbeitung (mdv) is developing journey planning systems and therefore has recently implemented the support of OSM data into their systems. A key aspect of journey planning is the georeferencing of the data in a GIS. In this talk the various aspects of railway and tramway routing will be presented. The intention is, to take the example of rail-bound routing and discuss the challenges as well as the benefits of OSM data for public transport and the implications for the different stakeholders: public transport providers, public transport users and the broad OpenStreetMap community.}



\abstractwithspeaker{11:30}{Building A}{Sven Geggus}{Map rendering beyond Mapnik}%
{A presentation of the two major FOSS alternatives: Mapserver and Geoserver}%
{Sven Geggus is long-time GIS and open source evangelist.}%
{Talking about Map rendering and tileservers in an environment of Openstreetmap activists makes it look like if there is only one piece of software which can be used to do this job: Mapnik.

This talk will focus on the two major FOSS alternatives, which are less well
known outside the FOSSGIS/OSGeo community: Geoserver and Mapserver

I will show some simple rendering examples for both of them (and Mapnik as a reference). Finally I will present advantages and disadvantages of each particular software.}

\abstractwithspeaker{12:00}{Building B}{Michael Collinson}{The State of the License }%
{The first two years of ODbL}%
{Mike has been a mapper since 2005. He served as Secretary on the the OpenStreetMap Foundation board and is currently Chair of the License Working Group. }%
{An overview about the work of the OSMF Licensing Working Group since the license change, and an opportunity to ask questions and discuss problems. }

\abstractwithspeaker{12:00}{Building A}{Stephan Bösch-Plepelits }{pgmapcss - Advanced cartography for Mapnik}%
{pgmapcss implements MapCSS for Mapnik - by moving the cartography into the database.}%
{}%
{pgmapcss combines MapCSS - a more or less standardized map description language - with Mapnik - a widely used map renderer for vector or bitmap maps. In contrast to other attempts at using CSS-like styling, the actual cartography process (evaluating which map features should be rendered  in the current view and how) is moved into a database function (Using PL/Python3 in a PostgreSQL database). Mapnik just needs to read the final properties - e.g. geometry, widths, colors, texts, \dots This database function is compiled by pgmapcss from the MapCSS style sheet.

\noindent There are many advantages to this idea:
\begin{itemize}
\item Writing MapCSS simplifies writing style sheets for Mapnik, as there's no longer a separation into database query and styling. Often those database queries were complex and required advanced knowledge of SQL. pgmapcss takes most of the complexity away. In addition, the actual database queries for map features are optimized for each zoom level.
\item MapCSS is powerful because map features can be related to each other (e.g. by relation membership or the proximity of objects). Furthermore calculations are possible (e.g. show the width of a line based on the value of a tag of the map feature). pgmapcss even offers geometric calculations (e.g. create lines between map features, rotation, buffers, …).
\item As there are more than ten libraries which implement MapCSS, the language is available on different platforms and for different output media. Styles can be used with various libraries,  although there are differences between dialects, but hopefully they will converge in the future.
\end{itemize}

Homepage: https://github.com/plepe/pgmapcss}


\abstractwithspeaker{14:00}{Building B}{Stefan Keller}{State of the Kort Game}%
{The First OpenStreetMap Mobile Mini Game Goes Public}%
{Stefan Keller is professor for information systems at the University for Applied Sciences (UAS/HSR) in Rapperswil (Switzerland). He leads the (OSGeo) Geometa Lab at Institute for Software. He teaches database management systems as well as geographic information systems (GIS) at Bachelor and Master level, as well as in advanced studies. Open Source and volunteered geographic information (VGI) play a role in many of his projects. }%
{Kort ist a mobile web app to fix OpenStreetMap data. It runs on the most common browsers. The app uses the concept of gamification. Game-like elements like points (so-called ``Koins'') are collected by the players by fulfilling a mission, like adding names to POIs without one. All proposed solutions are validated by other players. Once three players aggreed on a proposal, it is integrated on OpenStreetMap.

This is a report about our efforts to let volunteers contribute additional mission types. Besides a restructuring of mission types and internal refactoring issues, the main question is how to design the user and machine-oriented interface. (http://www.kort.ch)}


\abstractwithspeaker{14:00}{Building A}{Andy Allan}{Lightning Map Tiles}%
{A Vector Approach to Raster Maps}%
{Andy Allan created the OpenCycleMap map layer in 2007, and is now providing high-quality map tiles to hundreds of applications and websites through his Thunderforest mapping platform. He is also the author of the CartoCSS version of the ``Standard OpenStreetMap Style'' (see workshop on p. 43).}%
{Many of the maps you see created from OpenStreetMap data use the battle-hardened mod\_tile software stack to render images on-demand directly from a PostGIS database. To work around some of its inherent limitations, Andy started working on an alternative approach based on the mapnik-vector-tile software, storing the map data in protocol-buffer based vector tiles. This opens up a range of cartographic features, from faster map drawing times to easing the burden of hosting multiple styles. Andy will discuss the approaches taken, the pitfalls encountered and the future possibilities that this vector tile approach brings.}


\abstractwithspeaker{14:30}{Building B}{Dietmar Seifert}{OSM housenumber evaluation}%
{Improving address quantity and quality through automated analyses}%
{Dietmar Seifert has been an OSM contributor for about four years, active in the forum and best known in the German  community for his automated street list and housenumber reports.}%
{OpenStreetMap is set to cover more and more addresses. This talk demonstrates a tool designed to automatically detect missing house numbers, by comparing data from OSM with data retrieved from a city council or similar authority. The government data doesn't have to have a geo component---simple lists are sufficient. This approach has been used with good results in various places in Germany, and also includes a method to generate feedback to the administration (whose data is often less than perfect).}

\abstractwithspeaker{14:30}{Building A}{Konstantin Käfer}{Rendering Maps with OpenGL}%
{How to render a vast dataset like OSM on mobile devices}%
{After studying IT Systems Engineering in Potsdam, Konstantin Käfer joined Mapbox and now works mainly on map rendering.}%
{Rendering maps on the client becomes more important as technologies emerge and devices get faster. While most maps are still served as raster tiles, there is a case for delivering vectors to the client and having them rendered on the device. This talk will review existing rendering techniques and data formats, and point out their advantages and disadvantages.}

\abstractwithspeaker{15:00}{Building B}{Christian Quest}{State of the tools of OSM France}%
{OSM France develop and maintain a number of tools---let's take a tour}%
{Christian Quest, 48, lives near Paris. He has been an OSM contributor
since 2009, and he's a founding member and the current president of
OSM-FR. Focused on data quality and exhaustivity, working on several
renderings where external data are compared to OSM data to find "white
spots", missing data as well as OSM-FR styled tiles. Christian is also
part of OSM-FR tech team as system administrator taking care of a dozen
of servers. He discovered OSM through his other (past) activities:
paragliding and genealogy.}%
{Some tools available on the French servers are generic ones like Taginfo, a tile server or an instance of Overpass API. But some are our own developments, yet still useful for all. Among others we can find a HOT tile server, uMap, an API proxy, our cadastre extractor, a polygon generator, a generator for area extracts with diff, a boundary maker, the QA tool Osmose\dots}



\abstractwithspeaker{15:00}{Building A}{Brett Camper}{Tesselator’s Delight}%
{Introduction to OpenStreetMap for WebGL}%
{Brett Camper is interested in graphics programming, data visualization, game and interface design, and related areas. He recently helped start Mapzen, a company focused on developing tools and apps to improve open source geo.}%
{In the past couple years, WebGL has moved from an emerging technology to a widely supported one, providing a promising and powerful toolkit for rendering OSM data in the browser. But WebGL is often daunting and foreign to web programmers, and surprisingly few resources are available to learn.

In this talk I hope to help demystify WebGL for OSM, introducing the basics of a rendering pipeline using open source code examples, including: getting data from the server via GeoJSON or binary vector tiles, turning OSM geometries into triangle primitives ("tessellation"), constructing neatly joined polygonal line segments, extruding building outlines into 3D models, and creating lighting or perspective effects. All presented code is open source and also available as a Leaflet plugin, making it easily accessible to developers who want to experiment with adding WebGL and 3D components to their maps.}


\abstractwithspeaker{16:00}{Building B}{Frédéric Rodrigo}{Osmose}%
{A quality assurance tool for detecting and fixing errors and integrating OpenData}%
{Frédéric Rodrigo is secretary of OpenStreetMap France, a freelancer in geomatics, and works on quality tools and a couple other development projects around OSM.}%
{Osmose is one of many quality assurance tools available to detect errors and inconsistencies in OpenStreetMap data. It is also useful for integrating OpenData. Osmose has more than 250 different data checks, and the number of analyses is still rising. We're also rolling it out for more countries. With the latest funding, Osmose will also get an integrated tag editor usable on desktop and mobile.}

\abstractwithspeaker{16:00}{Building A}{Jan Marsch}{OSM Buildings}%
{Now and next}%
{Jan Marsch is from Berlin, Germany and has been software engineer for desktop and mobile web applications for about 16 years. By doing some projects for Nokia Maps five years ago, he got addicted to maps. His technical focus is on JavaScript, HTML5, Canvas, REST, PHP and databases. Other strenghts are entrepreneural thinking, performance optimization and user experience. He is happy working with small companies but also did long term projects for Here, Daimler, Bayer, Deutsche Lufthansa and TNT.}%
{I'll give you an introduction to OSM Buildings, a project to visualize OpenStreetMap building geometry in modern web browsers. You'll learn what its render modes are and how easily it integrates with existing web map engines.

If you are familiar with the project already, there are a lot new data sources to discover and user interaction has made a big step forward.

Furthermore I’ll explain how OSM Buildings compares to similar projects and where it stands out. For future plans, you'll see why there won't be another gray 3D blocks engine.

Meanwhile, follow @OSMBuildings on Twitter!}


\abstractwithspeaker{16:30}{Building B}{Serge Wroclawski}{MapRoulette 2: Electric Boogaloo}%
{It's back, it's real, and it's fun}%
{Serge Wroclawski is a longstanding member of the OpenStreetMap community. He's one of the founders of OSM US, and member of the OSMF Data Working Group member. He's also the author of ``Why the World Needs OpenStreetMap'', a blog post that has made headlines in early 2014.}%
{MapRoulette has been undergoing a major rewrite for nearly two years, but the new version is finally out. In this presentation, you will learn about MapRoulette's history and about what's new in the current version, as well as what the future may hold.}

\abstractwithspeaker{16:30}{Building A}{Vladimir Elistratov}{2.5D maps and bird views with Blender}%
{Creating 2.5D maps and bird views with OpenStreetMap and Blender}%
{Vladimir Elistratov works as a developer of web mapping applications. He an is OSM contributor since 2008. }%
{2.5D maps are ordinary 2D maps in the web Mercator projection enhanced with a layer of 3D buildings rendered in the oblique projection. A birdview is an attractive way to realistically represent neighborhood. Realistic 3D models of buildings are used in both cases.

This talk presents a method of using the Blender open source software to add 3D buildings to an OpenStreetMap map, capable of composing whole cities of 3D buildings. Different ways of adding these buildings to a Mapnik rendering are discussed. SRTM data is used to place buildings on terrain; if necessary, terrain data is edited manually in Blender. Another key problem for bird views is to develop attractive 3D representation for numerous street objects like trees, street lights, fences, benches, bus stops, etc.}

\abstractwithspeaker{17:00}{Building A}{Maxim Rylov }{Cartographically Plausible Label Placement}%
{A multi-criteria model for good point-label placement on OSM maps}%
{Maxim Rylov is a PhD student in GIScience Research Group, Department of Geography, Heidelberg University. His main research interests are digital and web cartography, automated label placement, computational geometry and GIS mapping.}%
{Cartographic label placement is a very important aspect of a map production. This task is essential for both traditional and automated cartography. Every OSM-based map is annotated using a labeling algorithm in one of the existing toolkits for rendering maps. However, none of these algorithms is able to take into account a rich set of well-established cartographic guidelines for feature annotation used by human cartographers. Assigning names to point-features is one of the map lettering tasks. In our talk we present an approach, expressed as a multi-
criteria optimization model, that complies with almost all well-defined cartographic placement principles and requirements for point-feature label placement. We show how this approach allows a significant increase in toponym density on an OSM-based map without effecting readability and legibility.}

